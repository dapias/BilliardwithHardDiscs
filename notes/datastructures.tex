% \documentclass[preprint]{revtex4-1}
\documentclass[11pt,letterpaper]{article}

\usepackage[utf8]{inputenc} \usepackage[spanish]{babel} \usepackage{amsmath}
\usepackage{amsfonts} \usepackage{amssymb} \usepackage{graphicx} \usepackage{hyperref}
\usepackage{mhchem} \usepackage{nameref} \usepackage{caption} \usepackage{subcaption}
\usepackage[left=1.5cm,top=1.5cm,right=1.5cm,bottom=2.5cm]{geometry} \usepackage{color}
\usepackage{epstopdf} \providecommand{\e}[1]{\ensuremath{\times 10^{#1}}}
\usepackage{wrapfig} \newcommand{\ncd}{\newcommand} \ncd{\mrm} {\mathrm} \ncd{\beq}
{\begin{equation}} \ncd{\eeq} {\end{equation}} \def\d{{\rm d}} \def\D{{\rm D}}
\def\f{{\rm f}} \def\g{{\rm g}} \def\r{\mathbf{r}} \def\p{\mathbf{p}} \def\q{\mathbf{q}}
\newcommand{\avg}[1]{\left< #1 \right>} % for average
\newtheorem{prop}{Proposición}[section]
\newtheorem{teor}[prop]{Teorema}


\begin{document}
\title{Importancia de las estructuras de datos en el cómputo científico}
\maketitle

\begin{abstract}
Estas notas están hechas con el objeto de describir la importancia de las estructuras de datos como herramientas necesarias para desarrollar código eficiente en el área del cómputo científico
\end{abstract}

\section{Generalidades}

\section{Array}

\section{Linked lists}

\section{Stacks, Queues y Deques}

\section{Priority Queues}


\end{document}
 























\end{document}
